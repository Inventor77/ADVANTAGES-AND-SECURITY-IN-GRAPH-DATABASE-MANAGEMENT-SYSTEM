\documentclass[12pt,a4paper]{article} 
\renewcommand{\baselinestretch}{1.5}
\usepackage[utf8]{inputenc}
\usepackage[letterspace=2]{microtype}
\usepackage{graphicx}

\begin{document}

%-------------------------------
    %	TITLE SECTION
%-------------------------------
    \topskip0pt
    \vspace*{\fill}
    \begin{center}
        \Huge
        \textbf{SHREEDUTT DIXIT}
        \\
        \Large
        (19111056)
        \\
        \textbf{BIOMEDICAL ENGINEERING}
        \\
        6th SEMESTER
        \\
        \Large
        \textbf{TERM PAPER TOPIC :}
        \\
        \large
        ADVANTAGES AND SECURITY IN GRAPH DATABASE MANAGEMENT SYSTEM
    \end{center}
    \vspace*{\fill}
    \pagebreak
    
    \hrule \medskip 
    \begin{center}
        \large 
    \textbf{TERM PAPER}\\
    \normalsize 
    \textbf{ADVANTAGES AND SECURITY IN GRAPH DATABASE MANAGEMENT SYSTEM}\\ 
    \end{center}
    
    \medskip\hrule
    \bigskip
    
    \begin{center}
    \subsection*{Introduction to Graph Database Management System}
    \end{center}
    
    A graph database is a single-purpose, specialised platform for designing and controlling graphs. Graphs are comprised of nodes, edges, and properties, which are all used to symbolise and hold data in a way that relational databases can't.
    \\
    Another commonly used term is graph analytics, which refers to the process of analysing data in a graph format with data points acting as nodes and relationships acting as edges.
    \\
    A database that can support graph formats is required for graph analytics; this can be a dedicated graph database or a converged database that supports multiple data models, including graph.
    \\
    \textbf{Property graphs} and \textbf{RDF} graphs are two popular graph database models.
    \\
    The property graph is more concerned with analytics and querying, whereas the RDF graph is more concerned with data integration. Both types of graphs are made up of a set of points (vertices) and the connections that connect them (edges).
\pagebreak

\rule{\textwidth}{1pt}
    \begin{center}
        \section* {Table Of Content}
    \end{center}
  \begin{itemize}
    \item Introduction to Graph Database Management System
    \item Property graphs
    \item RDF graphs
  \end{itemize}

  \rule{\textwidth}{1pt}
  
\pagebreak

Graph databases address large problems that many of us face on a daily basis. Modern data problems frequently involve many-to-many relationships with heterogeneous data, which necessitates the following:
\begin{itemize}
    \item Navigate deep hierarchies
    \item Discover hidden connections between distant items
    \item Find out how items are related to one another.
  \end{itemize}
Whether it's a social network, a payment network, or a road network, everything is an interconnected graph of relationships. And when we want to ask questions about the real world, many of them are about relationships rather than individual data elements.
\pagebreak

\subsubsection*{Property graphs}
Property graphs are used to model data relationships and enable query and data analytics based on these relationships. A property graph is made up of vertices that contain detailed information about a subject and edges that represent the relationship between the vertices. The vertices and edges can have attributes, known as properties, that are associated with them.
\\
Property graphs are used in a wide variety of industries and sectors due to their versatility, including finance, manufacturing, public safety, retail, and many others.
\pagebreak

\subsubsection*{RDF graphs}

RDF graphs (RDF stands for Resource Description Framework) adhere to a set of W3C (Worldwide Web Consortium) standards designed to represent statements and are best suited to representing complex metadata and master data. They are frequently used in the context of linked data, data integration, and knowledge graphs. They can represent complex domain concepts or provide rich semantics and inference on data.
\\
A statement is represented in the RDF model by three elements: two vertices connected by an edge that reflect the subject, predicate, and object of a sentence—this is known as an RDF triple. A unique URI, or Unique Resource Identifier, identifies each vertex and edge. The RDF model enables information exchange by allowing data to be published in a standard format with well-defined semantics. RDF graphs have been widely adopted by government statistics agencies, pharmaceutical companies, and healthcare organisations.

\pagebreak

\end{document}