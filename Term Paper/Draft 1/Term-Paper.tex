\documentclass[12pt,a4paper]{article} 
\renewcommand{\baselinestretch}{1.5}
\usepackage[utf8]{inputenc}
\usepackage[letterspace=2]{microtype}
\usepackage{graphicx}

\begin{document}

%-------------------------------
    %	TITLE SECTION
%-------------------------------
    \topskip0pt
    \vspace*{\fill}
    \begin{center}
        \Huge
        \textbf{SHREEDUTT DIXIT}
        \\
        \Large
        (19111056)
        \\
        \textbf{BIOMEDICAL ENGINEERING}
        \\
        6th SEMESTER
        \\
        \Large
        \textbf{TERM PAPER TOPIC :}
        \\
        \large
        ADVANTAGES AND SECURITY IN GRAPH DATABASE MANAGEMENT SYSTEM
    \end{center}
    \vspace*{\fill}
    \pagebreak
    
    \hrule \medskip 
    \begin{center}
        \large 
    \textbf{TERM PAPER}\\
    \normalsize 
    \textbf{ADVANTAGES AND SECURITY IN GRAPH DATABASE MANAGEMENT SYSTEM}\\ 
    \end{center}
    
    \medskip\hrule
    \bigskip
    
    \begin{center}
    \subsection*{Introduction to Graph Database Management System}
    \end{center}
    
    A graph database is a single-purpose, specialised platform for designing and controlling graphs. Graphs are comprised of nodes, edges, and properties, which are all used to symbolise and hold data in a way that relational databases can't.
    \\
    Another commonly used term is graph analytics, which refers to the process of analysing data in a graph format with data points acting as nodes and relationships acting as edges.
    \\
    A database that can support graph formats is required for graph analytics; this can be a dedicated graph database or a converged database that supports multiple data models, including graph.
    \\
    \textbf{Property graphs} and \textbf{RDF} graphs are two popular graph database models.
    \\
    The property graph is more concerned with analytics and querying, whereas the RDF graph is more concerned with data integration. Both types of graphs are made up of a set of points (vertices) and the connections that connect them (edges).
\pagebreak

\rule{\textwidth}{1pt}
    \begin{center}
    \section* {Table Of Content}
        \end{center}
  \begin{itemize}
    \item Introduction
  \end{itemize}

  \rule{\textwidth}{1pt}
  
\pagebreak


\end{document}